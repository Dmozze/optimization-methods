\documentclass[14pt,a4paper,report]{ncc}
\usepackage[a4paper, mag=1000, left=2.5cm, right=1cm, top=2cm, bottom=2cm, headsep=0.7cm, footskip=1cm]{geometry}
\usepackage[utf8]{inputenc}
\usepackage[english,russian]{babel}
\usepackage{indentfirst}
\usepackage[dvipsnames]{xcolor}
\usepackage[colorlinks]{hyperref}
\usepackage{listings}
\usepackage{pgfplotstable}
\usepackage{caption}
\usepackage{pgfplots}
\pgfplotsset{compat=1.9}
\usepackage{csvsimple}
\DeclareCaptionFont{white}{\color{white}}
\DeclareCaptionFormat{listing}{\colorbox{gray}{\parbox{\textwidth}{#1#2#3}}}
\captionsetup[lstlisting]{format=listing,labelfont=white,textfont=white}
\lstset{% Собственно настройки вида листинга
inputencoding=utf8, extendedchars=\true, keepspaces = true, % поддержка кириллицы и пробелов в комментариях
language=Pascal,            % выбор языка для подсветки (здесь это Pascal)
basicstyle=\small\sffamily, % размер и начертание шрифта для подсветки кода
numbers=left,               % где поставить нумерацию строк (слева\справа)
numberstyle=\tiny,          % размер шрифта для номеров строк
stepnumber=1,               % размер шага между двумя номерами строк
numbersep=5pt,              % как далеко отстоят номера строк от подсвечиваемого кода
backgroundcolor=\color{white}, % цвет фона подсветки - используем \usepackage{color}
showspaces=false,           % показывать или нет пробелы специальными отступами
showstringspaces=false,     % показывать или нет пробелы в строках
showtabs=false,             % показывать или нет табуляцию в строках
frame=single,               % рисовать рамку вокруг кода
tabsize=2,                  % размер табуляции по умолчанию равен 2 пробелам
captionpos=t,               % позиция заголовка вверху [t] или внизу [b]
breaklines=true,            % автоматически переносить строки (да\нет)
breakatwhitespace=false,    % переносить строки только если есть пробел
escapeinside={\%*}{*)}      % если нужно добавить комментарии в коде
}

\begin{document}
% Переоформление некоторых стандартных названий
\renewcommand{\chaptername}{Лабораторная работа}
\def\contentsname{Содержание}

% Оформление титульного листа
\begin{titlepage}
\begin{center}
\textsc{Университет ИТМО\\[5mm]
Факультет информационных технологий и программирования\\[2mm]
Направление прикладной математики и информатики}

\vfill

\textbf{ОТЧЁТ ПО ЛАБОРАТОРНЫМ РАБОТАМ\\[3mm]
курса «Методы оптимизации»\\[6mm]
Вариант 6
\\[20mm]
}
\end{center}

\hfill
\begin{minipage}{.45\textwidth}
Выполнили студенты:\\[2mm]
Мозжевилов Данил, Кучма Андрей\\
Группы: М3238, M3239\\[2mm]

\end{minipage}%
\vfill
\begin{center}
 Санкт-Петербург, \today
\end{center}
\end{titlepage}

% Содержание
\tableofcontents
\newpage

\chapter{Методы одномерной оптимизации}
\section{Исследование данного графика}

$f(x) = -5x^5 + 4x^4 - 12x^3 + 11x^2 - 2x + 1 \rightarrow min$ на промежутке $[-0.5, 0.5]$.

$$f'(x) = -25x^4 + 16x^3 - 36x^2 + 22x - 2$$
Производная на промежутке $[-0.5, 0.5]$ обращается в ноль только в одной точке $x_0 \approx 0.10986$, причем на промежутке $[-0.5, x_0]$ функция монотонно убывает и мнотонно возрастает на -- $[x_0, 0.5]$, следовательно данная функция является унимодальной на данном промежутке.
$$min(f(x)) = f(x_0) \approx 0.897633$$
Нахождение аналитического минимума у данной функции является весьма трудоемкой задачей, поэтомы мы возпользовались сторонним софтом, чтобы найти его примерное значение.

\begin{tikzpicture}
\begin{axis}[
        title = График функции f(x),
	    xlabel = {$x$},
	    ylabel = {$f(x)$},
	    domain = -0.5:0.5
    ]
    \addplot[blue] {-5*x^5 + 4 * x^4 - 12 * x^3 + 11 * x^2 - 2*x + 1};
\end{axis}
\end{tikzpicture}


\begin{tikzpicture}
\begin{axis}[
	table/col sep = semicolon,
	height = 0.6\paperheight,
	width = 0.65\paperwidth,
	xmin = 1,
	xmax = 12,
	/pgf/number format/1000 sep={},
	legend pos={north west}
]

\legend{
	$dichotomy$,
	$golden$,
	$fibonacci$,
	$parabola$
};

\addplot table [x={log}, y={cnt}]
              {tables/dichotomy.csv};
%косяк. не должно так быть(((((
\addplot table [x={log}, y={cnt}]
              {tables/golden.csv};
\addplot table [x={log}, y={cnt}]
              {tables/fibonacci.csv};
\addplot table [x={log}, y={cnt}]
              {tables/parabola.csv};			  
\end{axis}
\end{tikzpicture}

\begin{tikzpicture}
\begin{axis}[
	table/col sep = semicolon,
	height = 0.6\paperheight,
	width = 0.65\paperwidth,
	xmin = 1,
	xmax = 12,
	/pgf/number format/1000 sep={},
	legend pos={north west}
]

\legend{
	$dichotomy$,
	$fibonacci$,
};

\addplot table [x={log}, y={cnt}]
              {tables/dichotomy.csv};
\addplot table [x={log}, y={cnt}]
              {tables/fibonacci.csv};
\end{axis}
\end{tikzpicture}

\csvautotabular[separator=semicolon]{dichotomy1.csv}

\begin{tikzpicture}
\begin{axis}[
	table/col sep = semicolon,
	height = 0.6\paperheight,
	width = 0.65\paperwidth,
	xmin = 1,
	xmax = 12,
	/pgf/number format/1000 sep={},
	legend pos={north west}
]

\legend{
	$dichotomy$,
	$fibonacci$,
};

\addplot table [x={log}, y={cnt}]
              {tables/dichotomy.csv};
\addplot table [x={log}, y={cnt}]
              {tables/fibonacci.csv};
\end{axis}
\end{tikzpicture}



\begin{tikzpicture}
\begin{axis}[
	table/col sep = semicolon,
	height = 0.8\paperheight,
	width = 0.8\paperwidth,
	xmin = 1,
	xmax = 12,
	/pgf/number format/1000 sep={}
]

\addplot table [x={log}, y={cnt}]
              {tables/fibonacci.csv};
\end{axis}
\end{tikzpicture}

%\chapter{Методы многомерной оптимизации}


\section{Постановка задачи и цель работы}

\begin{enumerate}
    \item Реализовать алгоритмы:
    \begin{itemize}
        \item 	Метод градиентного спуска
        \item   Метод наискорейшего спуска
        \item   Метод сопряженных градиентов
    \end{itemize}
    Оценить как меняется скорость сходимости, если для поиска величины шага используются
    различные методы одномерного поиска.
    \item Проанализировать траектории методов для нескольких квадратичных
    функций: придумайте две-три квадратичные двумерные функции, на которых
    работа каждого из методов будет отличаться. Нарисовать графики с линиями
    уровня функций и траекториями методов.
    \item Исследовать, как зависит число итераций, необходимое методам для
    сходимости, от следующих двух параметров:
    \begin{itemize}
        \item числа обусловленности $k \geq 1$ оптимизируемой функции
        \item размерности пространства $n$ оптимизируемых переменных
    \end{itemize}
    Сгенерировать от заданных параметров $k$ и $n$ квадратичную задачу размерности $n$ с числом обусловленности
    $k$ и запустить на ней методы многомерной оптимизации с некоторой заданной точностью. Замерить число итераций $T(n, k)$, которое 
    потребовалось сделать методу до сходимости.
\end{enumerate}

\newpage
\section{Общая схема того, как мы реализовывали алгоритмы}

В начале мы создали классы Matrix, DiagonalMatrix и Vector и для них перегрузили операторы $'+'$, $'-'$, $'*'$ и $'[\ ]'$
 (класс DiagonalMatrix появился только под конец, когда мы уже начали тестировать и узнали, что для тестов нужны только 
 диагональные матрицы и оказалось, что в коде для матриц испльзовался только оператор $'*'$, поэтому мы не стали реализовывать остальные 
 перегрузки для этого класса).

 Далее мы решили не использовать лямда-функции для задания квадратичных форм, а сделать отдельные классы
QuadraticFunction и \newline DiagonalQuadraticFunction, в которых храниться матрица $A$, вектор $b$ и число $c$, и просто
передавать их в качестве параметров в реализуемые алгоритмы, к тому же в классе можно хранить всю историю
 обращения к функции, что мы и делали.

Также мы создали класс GeneratorQudraticFunction, который генерировал рандомые
 вектора по заданной размерности и числу обусловленности.

 Пример того, как выглядели наши сгенерированные функции:
\newline

\texttt{\small 1 38.0198 208.636 276.712 419.618 517.318 549.321 565.029 598.464 641 }

\texttt{\small 1 56.0696 86.0772 94.8904 129.966 133.73 151.615 295.072 304.457 330.866 }

\texttt{\small 1 121.07 250.754 316.186 452.644 463.517 492.598 526.129 690.467 706.2} 
\newline

Первая строчка это диагональная матрица $A$. В данном случае с числом обусловленноости $641$.
Вторая строчка это вектор $b$. Прибавление константы мы решили не генерировать, так как на поиск точки
минимума она не влияет. Третья строка это начальное приближение $x_0$. 

Все сгенерированные функции мы сохраняли в файлы,
и благодаря этому не приходилось заново генерировать функции для каждого запуска программы, а также была
возможность запускать тестирование на каждом методе отдельно. 

Точность для алгоритмов мы решили задать всего лишь $0.01$, так как при тестировании не хотелось ждать по 30 минут, пока
алгоритмы найдут необходимый минимум для всех сгенерированных функций. Да и это не требовалось, так как судя по
данным, которые мы получали, этой точности хватало, чтобы получать приближение до пяти знаков после запятой.
 Также при вычислении минимума 
у функции размерности
$n = 10^4$ пришлось ограничится числом обусловленности до $k = 1000$, так как наши алгоритмы работали очень долго.

Отрисовывали графики мы средствами ЯП Python. Основная библиотека, которая использовалась
 — matplotlib. Была написана программа, которая по трём выходным .csv от основной программы, генерировала
  код и исполняла его. Так же в целях получать более понятные графики
 отрисовывание векторов и кривых уровня делалось только для первых 15 входных точек.

\newpage
\section{Иллуюстрации работы градиентных методов на двумерных квадратичных функцкиях}

% Для отрисовки некоторых функций нам пришлось ограничить число итераций, так как 
% в некоторых случаях были большие нагромождения. Например так

% \includegraphics{4566.jpg}

\subsection{Первая функция}
Рассмотрим функцию $$f(x, y) = x^2 - xy + 4y^2 + 2x + y$$. В матричном виде ее вид 
$f(x) = 1/2 * (Ax, x) + b * x$, где $A = $
\begin{pmatrix}
    2 & -1\\
    -1 & 8
\end{pmatrix}
и $b = $
\begin{pmatrix}
    2 \\
    1
\end{pmatrix}.

$det(A - \lambda E) = $
\begin{vmatrix}
    2 - \lambda & -1\\
    -1 & 8 - \lambda
\end{vmatrix}
$ = (2 - \lambda) * (8 - \lambda) - 1 = 15 - 10 * \lambda + \lambda^2 = (5 + \sqrt{10} - \lambda) * (5 - \sqrt{10} - \lambda)$.
Собственные значение матрицы $A$ положительны, следовательно квадратичная форма $f$ положительно определенная, а 
следовательно выпукла вниз. Число обусловленности $k = \frac{5 + \sqrt{10}}{5 - \sqrt{10}} \approx 4.4415$.
Для начала найдем точку минимума функции аналитически.


Надем точку, в которой градиент данной функции обращается в ноль. Это и будет точка минимума функции.
$grad\ f = $
\begin{pmatrix}
    2 * x - y + 2 & -x + 8y + 1
\end{pmatrix}$^T = (0\ 0)^T$.
Решив систему линейных уравнений, получаем 
$$x_m = -\frac{17}{15},\ y_m = -\frac{4}{15},\ f(x_m, y_m) = -\frac{19}{15}$$

Теперь покажем как наши методы находили минимум этой функции.
Начальную точку для всех методов мы взяли $x_0 = (-1, 10)$

\begin{enumerate}
    \item Метод градиентного спуска
    
    \includegraphics{gradient1.jpg}

    Как мы видим у этой функции происходят биения.

    Всего в алгоритме произошло 35 итераций. Найденная точка минимума
    $x_m = [-1.1333329668, -0.26666772037]$.

    \item Метод наискорейшего спуска
    
    \includegraphics{steepest1.jpg}

    Мы увеличили масштаб, так как большинство итераций происходило
    около минимума.

    Итераций произошло больше, чем в предыдущем методе -- 63.
    Найденная точка минимума $x_m = [-1.1333273611, -0.2666656975]$.
    Также заметим, что вектора не касаются линий уровня, мы это объясним ниже.

    \item Метод сопряженных градиентов

    \includegraphics{conjugate1.jpg}

    Всего итераций 2.
    Найденная точка минимума \newline $x_m = [-1.1333333333, -0.26666666667]$.

\end{enumerate}

Также общая картинка для всех методов.

\includegraphics{all1.jpg}


 \subsection{Вторая функция}
Рассмотрим вторую функцию для исследования $$f(x, y) = 8x^2 + 5y^2 + 10xy + 5x + 6y$$
Аналогично предыдущей функции находим собственные значения, убедимся, что они положительны, находим число
обусловленности, находим точку минимума
и само значение минимума.
$$k \approx 9.1574$$
$$x_m = \frac{1}{6},\ y_m = -\frac{23}{30}$$
$$f(x_m, y_m) = -\frac{133}{60}$$

Работа алгоритмов. Начальная точка $x_0 = (-5, 7)$
\newpage
\begin{enumerate}
    \item Метод градиентного спуска
    
    \includegraphics{gradient2.jpg}

    Как мы видим в начале происходит значительно меньше биений, но число
    итераций все равно больше -- 68, из-за того, что в конце их было много
    (этого нет на рисунке, так как мы ограничили число векторов).
    Найденная точка минимума $x_m = [0.16666493859, -0.76666406897]$.



    \item Метод наискорейшего спуска

    \includegraphics{steepest2.jpg}

    Всего итераций $67$. В конце также присутствуют биения. 
    И найденая точка минимума \newline $x_m = [0.16666492721, -0.7666635937]$.
    Также отметим, что в отличие от предыдущей функции, уже есть касание линий уровня у большинства векторов.
    \newpage
    \item Метод сопряженных градиентов
    
    \includegraphics{conjugate2.jpg}

    Как и на прошлой функции всего 2 итерации. Найденная точка минимума $x_m = [0.16666666667, -0.76666666667]$.

\end{enumerate}

Также все методы на одной картинке.

\includegraphics{all2.jpg}

\subsection{Третья функция}
Ну и наконец третья функция $$f(x, y) = 22x^2 + 2y^2 + xy + x + y$$
Аналогично с предыдущими
$$k \approx 11.07542$$
$$x_m = -\frac{3}{175},\ y_m = -\frac{43}{175}$$
$$f(x_m, y_m) = -\frac{23}{175}$$

Началная точка $x_0 = (13, 1)$.


\begin{enumerate}
    \item Метод градиентного спуска
    
    Как мы видим в отличие от двух предыдущих функций, если у первой происходили бения только в начале,а у
    второй только в конце, то тут биения происходят на протежении всего алгоритма, итого итераций -- 99.
    Найденная точка минимума $x_m = [-0.017143071323, -0.24571427597]$.



    \item Метод наискорейшего спуска
    

    Всего итераций $67$ и также большинство итераций произошло в конце.
    Найденная точка минимума $x_m = [-0.017142696872, -0.24571224089]$.

    \item Метод сопряженных градиентов
    
    Аналогично предыдущим методам 2 итреции.
    Найденная точка минимума $x_m = [-0.017142857143, -0.24571428571]$.



\end{enumerate}

Также все методы на одной картинке.

\includegraphics{all3.jpg}

\newpage
\section{Метод градиентного спуска}

Заметим, что в методе градиентного спуска константа линейной скорости сходимости $q = 2/(l + L)$ не зависит
от размерности пространства $n$, а только от собственных чисел матрицы $A$ квадратичной формы, а следовательно
для всех размерностей должны получится схожие результаты, что мы как раз таки видим на графике ниже.
Но есть несколько минимумов, которые выбиваются из общей массы.
 Скорее всего это из-за того, что сгенерированная
точка попала в многомерный овраг и из-за этого не происходило сильных биений. По сгенерированным
тестам это сложно понять.

На втором графике также видно, что размерность мало на что вилияет, нет явной закономерности.


\begin{flushleft}
    \begin{tikzpicture}
        \begin{axis}[
            table/col sep = semicolon,
            xlabel = {$k$},
            ylabel = {$times$},
            height = 0.55\paperheight,
            width = 0.8\paperwidth,
            /pgf/number format/1000 sep={},
            legend pos={north west},
            grid=major,
        ]
        
        \legend{
            $n = 10$,
            $n = 10^2$,
            $n = 10^3$,
            $n = 10^4$,
        };

        \addplot table [x={k}, y={times}]
                  {descent/10k.csv};
        \addplot table [x={k}, y={times}]
                  {descent/100k.csv};
        \addplot table [x={k}, y={times}]
                  {descent/1000k.csv};
        \addplot table [x={k}, y={times}]
                  {descent/10000k.csv};
        \end{axis}
    \end{tikzpicture}
\end{flushleft}


\begin{flushleft}
    \begin{tikzpicture}
        \begin{axis}[
            table/col sep = semicolon,
            xlabel = {$n$},
            ylabel = {$times$},
            height = 0.35\paperheight,
            width = 0.8\paperwidth,
            /pgf/number format/1000 sep={},
            legend pos={north west},
            grid=major,
        ]
        
        \addplot table [x={n}, y={times}]
                      {descent/n_good.csv};
        \end{axis}
    \end{tikzpicture}
\end{flushleft}

\newpage

\section{Метод наискорейшего спуска}

В этом методе получилось практически также как и в предыдущем. Размерность влияет на число итераций,
но не значительно. Скорее всего это связано с тем, что константа суперлинейной сходимости в зависимости от
размерности стремится к нулю, и поэтому для малых размерностей она стремится быстрее, чем для больших.

Промежуток одномерной оптимизации мырешили взять $(0, 0.1)$, так как мы так и не поняли
какой именно нужно брать, но мы провели несколько эксперементов подбирая правое значения и увидели,
что в очень редких случаях искомый минимум находится правее $0.1$. Также из-за этого в одном из примеров выше 
получилось так, что вектора не касались линий уровня.

Для одномерной оптимизации мы использовали метод Брента, как самый быстрый метод.

Точность для одномерной оптимизации решили взять $epsion^2$, где $epsion = 0.01$ точность, которая у нас была
задана для всех методов многомерной оптимизации. Решили так сделать, так как если передавать $epsilon$ без
изменений, то одномерная оптимизация зачастую даже не запускалась или делала всего одно вычисление функции,
так как $\alpha^*$ было меньше, чем $0.01$.

\begin{flushleft}
    \begin{tikzpicture}
        \begin{axis}[
            table/col sep = semicolon,
            xlabel = {$k$},
            ylabel = {$times$},
            height = 0.4\paperheight,
            width = 0.8\paperwidth,
            /pgf/number format/1000 sep={},
            legend pos={north west},
            grid=major,
        ]
        
        \legend{
            $n = 10$,
            $n = 10^2$,
            $n = 10^3$,
            $n = 10^4$,
        };

        \addplot table [x={k}, y={times}]
                  {steepest/10k.csv};
        \addplot table [x={k}, y={times}]
                  {steepest/100k.csv};
        \addplot table [x={k}, y={times}]
                  {steepest/1000k.csv};
        \addplot table [x={k}, y={times}]
                  {steepest/10000k.csv};
        \end{axis}
    \end{tikzpicture}
\end{flushleft}


\begin{flushleft}
    \begin{tikzpicture}
        \begin{axis}[
            table/col sep = semicolon,
            xlabel = {$n$},
            ylabel = {$times$},
            height = 0.45\paperheight,
            width = 0.8\paperwidth,
            /pgf/number format/1000 sep={},
            legend pos={north west},
            grid=major,
        ]
        
        \addplot table [x={n}, y={times}]
                      {steepest/n_good.csv};
        \end{axis}
    \end{tikzpicture}
\end{flushleft}


\newpage
\section{Метод сопряженных градиентов}

Для этого метода решили не ограничиваться числом обусловленности \newline $k = 1000$ для размерности $n = 10000$,
так как число итераций у этого метода значительно меньше, чем у остальных.

Этот метод не может иметь число итераций больше, чем размерность пространства $n$, так как попарно ортогональных
векторов в пространстве размерности $n$ есть ровно $n$, а в этом методе мы как раз так выбираем текущий вектор таким
образом, что он ортогонален всем предыдущим. Мы как раз и наблюдаем такую тенденцию на наших графиках.
Также заметим, что число обусловленности вляет на число итераций только до некоторого момента, а после это 
число просто колебается вокруг какого-то одного значения.

\begin{flushleft}
    \begin{tikzpicture}
        \begin{axis}[
            table/col sep = semicolon,
            xlabel = {$k$},
            ylabel = {$times$},
            height = 0.6\paperheight,
            width = 0.8\paperwidth,
            /pgf/number format/1000 sep={},
            legend pos={north west},
            grid=major,
        ]
        
        \legend{
            $n = 10$,
            $n = 10^2$,
            $n = 10^3$,
            $n = 10^4$,
        };

        \addplot table [x={k}, y={times}]
                  {conjugate/10k.csv};
        \addplot table [x={k}, y={times}]
                  {conjugate/100k.csv};
        \addplot table [x={k}, y={times}]
                  {conjugate/1000k.csv};
        \addplot table [x={k}, y={times}]
                  {conjugate/10000k.csv};
        \end{axis}
    \end{tikzpicture}
\end{flushleft}


\begin{flushleft}
    \begin{tikzpicture}
        \begin{axis}[
            table/col sep = semicolon,
            xlabel = {$n$},
            ylabel = {$times$},
            height = 0.6\paperheight,
            width = 0.8\paperwidth,
            /pgf/number format/1000 sep={},
            legend pos={north west},
            grid=major,
        ]
        
        \addplot table [x={n}, y={times}]
                      {conjugate/n_good.csv};
        \end{axis}
    \end{tikzpicture}
\end{flushleft}

\newpage
\section{Код реализованных методов}

\subsection{Метод градиентного спуска}
QuadraticFunctionType -- это либо обычная квадратичная функция,
 либо функция у которой основная матрица имеет диагональный вид.

 Возвращаем только число итераций, так как вся информация о вычислении
 функции остается внутри класса QuadraticFunctionType.

\begin{lstlisting}
size_t gradient_descent(QuadraticFunctionType &function,
                    Vector &x0, T alpha) const {
    Vector x = x0;
    T f_x = function.calc(x);
    size_t cnt = 0;
    while (true) {
        Vector gradient = function.gradient(x);
        T norma_gradient = gradient.norma();
        if (norma_gradient < epsilon) {
            break;
        }
        cnt++;
        while (true) {
            Vector gradient_prod_alpha = gradient * alpha;
            Vector y = x - gradient_prod_alpha;
            T f_y = function.calc(y);
            if (f_y < f_x) {
                x = y;
                f_x = f_y;
                break;
            }
            alpha /= 2;
        }
    }
    return cnt;
}
\end{lstlisting}

\newpage


\subsection{Метод наискорейшего спуска}

\begin{lstlisting}
size_t steepest_descent(QuadraticFunctionType &function,
                    Vector x0) const {
    // Инициализирую методы одномерной оптимизации
    search_methods searchMethods(epsilon * epsilon); 
    size_t cnt = 0;
    while (true) {
        Vector gradient = function.gradient(x0);
        if (gradient.norma() < epsilon) {
            break;
        }
        cnt++;

        // Лямбда-функция для одномерной оптимизации
        std::function<T(T)> F = [&](T alpha) {
            Vector grad_alpha = gradient * alpha;
            Vector calc_vec = x0 - grad_alpha;
            return function.calc_without_history(calc_vec);
        };

        range rang(0, 0.1L);
        information_search informationSearch = searchMethods.combined_brent(F, rang);
        T alpha_min = informationSearch.point;
        Vector gradient_alpha_min = gradient * alpha_min;
        function.calc(x0);
        x0 = x0 - gradient_alpha_min;
    }
    return cnt;
}


\end{lstlisting}


\newpage
\subsection{Метод сопряженных градиентов}


\begin{lstlisting}
size_t conjugate_gradient(QuadraticFunctionType &function, Vector &x0) const {
    std::vector<Vector> p;
    Vector gradient0 = function.gradient(x0);
    p.emplace_back(gradient0 * (-1.0L));

    std::vector<Vector> x;
    x.emplace_back(x0);
    function.calc(x.back());

    Vector last_A_pk(x0.size());
    size_t cnt = 0;
    while (true) {
        if (gradient0.norma() < epsilon) {
            break;
        }
        cnt++;
        Vector A_pk = function.hessian() * p.back();
        T A_pk_prod_pk = A_pk * p.back();
        T grad_norma_2 = gradient0 * gradient0;
        T alpha_k = grad_norma_2 / A_pk_prod_pk;

        Vector alpha_pk = p.back() * alpha_k;
        x.emplace_back(x.back() + alpha_pk);

        //Данное вычисление функции не влияет на ход алгоритма.
        //Это нужно только для статики.
        function.calc(x.back());

        Vector alpha_k_A_pk = A_pk * alpha_k;
        Vector gradient1 = gradient0 + alpha_k_A_pk;

        T bk = (gradient1 * gradient1) / (gradient0 * gradient0);

        Vector bk_pk = p.back() * bk;
        Vector antigradient = gradient1 * (-1.0L);
        p.emplace_back(antigradient + bk_pk);

        gradient0 = gradient1;
    }
    return cnt;
}

\end{lstlisting}

 Отчёт к каждой работе оформляется в отдельном файле

\end{document}