\chapter{Методы одномерной оптимизации}
\section{Исследование данного графика}

$f(x) = -5x^5 + 4x^4 - 12x^3 + 11x^2 - 2x + 1 \rightarrow min$ на промежутке $[-0.5, 0.5]$.

$$f'(x) = -25x^4 + 16x^3 - 36x^2 + 22x - 2$$
Производная на промежутке $[-0.5, 0.5]$ обращается в ноль только в одной точке $x_0 \approx 0.10986$, причем на промежутке $[-0.5, x_0]$ функция монотонно убывает и мнотонно возрастает на -- $[x_0, 0.5]$, следовательно данная функция является унимодальной на данном промежутке.
$$min(f(x)) = f(x_0) \approx 0.897633$$
Нахождение аналитического минимума у данной функции является весьма трудоемкой задачей, поэтомы мы возпользовались сторонним софтом, чтобы найти его примерное значение.

\begin{tikzpicture}
\begin{axis}[
        title = График функции f(x),
	    xlabel = {$x$},
	    ylabel = {$f(x)$},
	    domain = -0.5:0.5
    ]
    \addplot[blue] {-5*x^5 + 4 * x^4 - 12 * x^3 + 11 * x^2 - 2*x + 1};
\end{axis}
\end{tikzpicture}


\begin{tikzpicture}
\begin{axis}[
	table/col sep = semicolon,
	height = 0.6\paperheight,
	width = 0.65\paperwidth,
	xmin = 1,
	xmax = 12,
	/pgf/number format/1000 sep={},
	legend pos={north west}
]

\legend{
	$dichotomy$,
	$golden$,
	$fibonacci$,
	$parabola$
};

\addplot table [x={log}, y={cnt}]
              {tables/dichotomy.csv};
%косяк. не должно так быть(((((
\addplot table [x={log}, y={cnt}]
              {tables/golden.csv};
\addplot table [x={log}, y={cnt}]
              {tables/fibonacci.csv};
\addplot table [x={log}, y={cnt}]
              {tables/parabola.csv};			  
\end{axis}
\end{tikzpicture}

\begin{tikzpicture}
\begin{axis}[
	table/col sep = semicolon,
	height = 0.6\paperheight,
	width = 0.65\paperwidth,
	xmin = 1,
	xmax = 12,
	/pgf/number format/1000 sep={},
	legend pos={north west}
]

\legend{
	$dichotomy$,
	$fibonacci$,
	$golden$
};

\addplot table [x={log}, y={cnt}]
              {tables/dichotomy.csv};
\addplot table [x={log}, y={cnt}]
              {tables/fibonacci.csv};
\addplot table [x={log}, y={cnt}]
              {tables/golden.csv};
\end{axis}
\end{tikzpicture}

%\csvautotabular[separator=semicolon]{table/dichotomy.csv}


\begin{tikzpicture}
\begin{axis}[
	table/col sep = semicolon,
	height = 0.8\paperheight,
	width = 0.8\paperwidth,
	xmin = 1,
	xmax = 12,
	/pgf/number format/1000 sep={}
]

\addplot table [x={log}, y={cnt}]
              {tables/fibonacci.csv};
\end{axis}
\end{tikzpicture}
