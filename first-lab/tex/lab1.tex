\chapter{Методы одномерной оптимизации}
\section{Постановка задачи}
Реализовать алгоритмы одномерной оптимизации:

\begin{itemize}
	\item 	Метод дихотомии
	\item 	Метод золотого сечения
	\item	Метод Фибоначчи
	\item 	Метод парабол
	\item 	Комбинированный метод Брента
\end{itemize}

Протестировать реализованные алгоритмы на данной функции:
$$f(x) = -5x^5 + 4x^4 - 12x^3 + 11x^2 - 2x + 1 \rightarrow min$$ на промежутке $[-0.5, 0.5]$.

\section{Исследование данного графика}

$f(x) = -5x^5 + 4x^4 - 12x^3 + 11x^2 - 2x + 1 \rightarrow min$ на промежутке $[-0.5, 0.5]$.

$$f'(x) = -25x^4 + 16x^3 - 36x^2 + 22x - 2$$
Производная на промежутке $[-0.5, 0.5]$ обращается в ноль только в одной точке $x_0 \approx 0.10986$, причем на промежутке $[-0.5, x_0]$ функция монотонно убывает и мнотонно возрастает на -- $[x_0, 0.5]$, следовательно данная функция является унимодальной на данном промежутке.
$$min(f(x)) = f(x_0) \approx 0.897633$$
Нахождение аналитического минимума у данной функции является весьма трудоемкой задачей, поэтомы мы возпользовались сторонним софтом, чтобы найти его примерное значение.

Также можно заментить, что возрастающая часть графика возрастает довльно медленно и скорее всего 
метод парабол будет значительно уступать остальным методам.


\begin{tikzpicture}
\begin{axis}[
        title = График функции f(x),
	    xlabel = {$x$},
	    ylabel = {$f(x)$},
        height = 0.8\paperheight,
	    width = 0.65\paperwidth,
	    domain = -0.5:0.5
    ]
    \addplot[blue] {-5*x^5 + 4 * x^4 - 12 * x^3 + 11 * x^2 - 2*x + 1};
\end{axis}
\end{tikzpicture}
\newpage
\section {Таблицы}
	Все таблицы предоставлены при точности $epsilon = 1e\!\!-\!\!11$
\subsection {Метод дихотомии}

\resizebox{12cm}{!}{
\csvautotabular[separator=semicolon]{tables/dichotomy1e-11.csv}
}

\ 

Этот метод хотя и имеет меньшее количество итераций, чем отстальные,
но на каждой итерации происходит два вызова функции, что очень дорого.

\subsection {Метод золотого сечения}

\resizebox{12cm}{!}{
\csvautotabular[separator=semicolon]{tables/golden1e-11.csv}
}

\subsection {Метод Фибоначчи}

\resizebox{12cm}{!}{
\csvautotabular[separator=semicolon]{tables/fibonacci1e-11.csv}
}

\ 

Примерно с 30 итерации заметны какие либо отличия от метода золотого сечения, 
но значительные отличия заметны только к концу, когда числа 
Фибоначчи не большие.

\subsection {Метод парабол}

\resizebox{12cm}{!}{
\csvautotabular[separator=semicolon]{tables/parabola1e-11.csv}
}

\ 

Как мы видим, левая граница долгое время не менялась.
Это из-за того, что наша функция в возрастающей части
возрастает довольно медленно, и из-за этого точка минимума параболы
не далеко отходит от правой границы отрезков.

\subsection {Комбинированный метод Брента}

\resizebox{12cm}{!}{
\csvautotabular[separator=semicolon]{tables/brent1e-11.csv}
}

\ 

Как мы видим во многих местах отношение длин отрезков
близко к золотому сечению, что и компенсирует то, что данная
функция возрастает довольно медленно.

\section {Исследование метода парабол при различных начальных приближениях}

\begin{tikzpicture}
    \begin{axis}[
        table/col sep = semicolon,
        xlabel = {$x$},
        ylabel = {$num\ cnt$},
        height = 0.6\paperheight,
        width = 0.65\paperwidth,
        /pgf/number format/1000 sep={},
        legend pos={north west}
    ]
    
    \addplot table [x={point}, y={cnt}]
                  {tables/research_parabolas1e-11.csv};
    \end{axis}
\end{tikzpicture}


\section {Графики}


\begin{tikzpicture}
\begin{axis}[
	table/col sep = semicolon,
	xlabel = {$-log10(eps)$},
	ylabel = {$num\ cnt$},
	height = 0.6\paperheight,
	width = 0.65\paperwidth,
	xmin = 1,
	xmax = 12,
	/pgf/number format/1000 sep={},
	legend pos={north west}
]

\legend{
	$dichotomy$,
	$golden$,
	$fibonacci$,
	$parabola$,
	$brent$
};

\addplot table [x={log}, y={cnt}]
              {tables/dichotomy.csv};
\addplot table [x={log}, y={cnt}]
              {tables/golden.csv};
\addplot table [x={log}, y={cnt}]
              {tables/fibonacci.csv};
\addplot table [x={log}, y={cnt}]
              {tables/parabola.csv};
\addplot table [x={log}, y={cnt}]
              {tables/brent.csv};

\end{axis}
\end{tikzpicture}

Как и предполагалось метод парабол показал худшие результаты.
Хотя и в некоторых точках у нас и есть результаты, которые
не такие прохие, но это из-за того, что в нашей реализации
мы подбираем начальное распределение рандомом.
По графикам видно, что метод дихотомии значительно уступает
методам золотого сечения, фибоначчи и комбинированному методу Брента.
Также заметим, что метод золотого сечения практически не отличается от метода
Фибоначчи. Я думаю это связано с тем, что если посмотреть на таблицы
по этим двум мотодам, то можно заметить, что отношения
длин отрезков на соседних итерациях начинают значительно различаться
только в конце, т. е. при малых числах Фибоначчи.

\newpage
\section {Тестирование на многомодальной функции}
Мы взяли функцию 
$f(x) = x * sin(x^2)$
и находим ее минимум на промежутке $[0, 3.7]$
Все наши алгоритмы находят минимум $min_a (f(x)) = -2.183...$
хотя глобальный минимум равен $min(f(x)) = -3.319...$.

Графики:

\ 


\includegraphics{Screenshot from 2021-03-10 10-53-01.png}

\begin{tikzpicture}
	\begin{axis}[
		table/col sep = semicolon,
		xlabel = {$-log10(eps)$},
		ylabel = {$num\ cnt$},
		height = 0.6\paperheight,
		width = 0.65\paperwidth,
		xmin = 1,
		xmax = 12,
		/pgf/number format/1000 sep={},
		legend pos={north west}
	]
	
	\legend{
		$dichotomy$,
		$golden$,
		$fibonacci$,
		$parabola$,
		$brent$
	};
	
	\addplot table [x={log}, y={cnt}]
				  {tables/dichotomyB.csv};
	\addplot table [x={log}, y={cnt}]
				  {tables/goldenB.csv};
	\addplot table [x={log}, y={cnt}]
				  {tables/fibonacciB.csv};
	\addplot table [x={log}, y={cnt}]
				  {tables/parabolaB.csv};
	\addplot table [x={log}, y={cnt}]
				  {tables/brentB.csv};
	
	\end{axis}
	\end{tikzpicture}


\newpage
\section {Код}

\subsection {Вспомогательный код}

\begin{lstlisting}
class range {
    std::vector<std::pair<long double, long double>> range_history;
public:

    range(long double l, long double r) : range_history() {
        range_history.emplace_back(l, r);
    }
    long double delta() const {
        return right() - left();
    }
    long double median() const {
        return (right() + left()) / 2;
    }
    long double right() const {
        return range_history.back().second;
    }
    long double left() const {
        return range_history.back().first;
    }
    void update_range(long double l, long double r) {
        range_history.emplace_back(l, r);
    }
    void print(std::ofstream &out, std::function<long double(long double)> &func) {
        out << "iter;left;right;ratio;f(left);f(right)" << std::endl;
        for (size_t i = 0; i < range_history.size(); i++) {
            out << i + 1 << ';';
            const long double l = range_history[i].first;
            const long double r = range_history[i].second;
            out << l << ';' << r << ';';
            if (i > 0) {
                const long double l_ = range_history[i - 1].first;
                const long double r_ = range_history[i - 1].second;
                out << (r_ - l_) / (r - l);
            }
            out << ';' << func(l) << ';' << func(r) << std::endl;
        }
    }
    void print(std::ostream &out, std::function<long double(long double)> &func) {
        out << "iter;left;right;ratio;f(left);f(right)" << std::endl;
        for (size_t i = 0; i < range_history.size(); i++) {
            out << i + 1 << ';';
            const long double l = range_history[i].first;
            const long double r = range_history[i].second;
            out << l << ';' << r << ';';
            if (i > 0) {
                const long double l_ = range_history[i - 1].first;
                const long double r_ = range_history[i - 1].second;
                out << (r_ - l_) / (r - l);
            }
            out << ';' << func(l) << ';' << func(r) << std::endl;
        }
    }
};
\end{lstlisting}

\begin{lstlisting}
struct information_search {
    long double point;
    long double value;
    int times;
    range r;
    information_search(long double point, long double value, int times, range &r)
    :
    point(point),
    value(value),
    times(times),
    r(r) {}
};
\end{lstlisting}

\begin{lstlisting}
class search_methods {
    std::vector<long double> f;
    const size_t number_fib = 90;

    long double epsilon;

    const long double golden_const = 0.5 * (sqrtl(5) - 1);

    long double F(size_t n);

    static long double calc_u(const long double &x1, const long double &f_x1,
                       const long double &x2, const long double &f_x2,
                       const long double &x3, const long double &f_x3);
    
    static std::function<long double(long double)>
    find_cnt_func(std::function<long double(long double)> &func, size_t &cnt);
public:

    explicit search_methods(long double epsilon) : epsilon(epsilon), f() {
        f.push_back(1.);
        f.push_back(1.);
        for (int i = 0; i < number_fib; i++) {
            f.push_back(f.back() + f[f.size() - 2]);
        }
    }

    information_search dichotomy(std::function<long double(long double)> &func, range r) const;

    information_search golden_ratio(std::function<long double(long double)> &func, range r) const;

    information_search fibonacci(std::function<long double(long double)> &func, range r);

    information_search parabolas(std::function<long double(long double)> &func, range r) const;

    information_search combined_brent(std::function<long double(long double)> &func, range r) const;
};
\end{lstlisting}

\subsection {Метод дихотомии}

\begin{lstlisting}
information_search search_methods::dichotomy(std::function<long double(long double)> &func, range r) const {
	size_t cnt = 0;
	std::function<long double(long double)> func_cnt = find_cnt_func(func, cnt);
	long double delta = epsilon / 2;
	while (r.delta() / 2 > epsilon) {
		long double x1 = (r.right() + r.left() - delta) / 2;
		long double x2 = (r.right() + r.left() + delta) / 2;
		if (func_cnt(x1) <= func_cnt(x2)) {
			r.update_range(r.left(), x2);
		} else {
			r.update_range(x1, r.right());
		}
	}
	information_search answer(r.median(), func_cnt(r.median()), cnt, r);
	return answer;
}
\end{lstlisting}

\newpage
\subsection {Метод золотого сечения}

\begin{lstlisting}
information_search search_methods::golden_ratio(std::function<long double(long double)> &func, range r) const {
    size_t cnt;
    std::function<long double(long double)> func_cnt = find_cnt_func(func, cnt);
    long double x1 = r.right() - golden_const * r.delta();
    long double x2 = r.left() + golden_const * r.delta();
    long double f_x1 = func_cnt(x1);
    long double f_x2 = func_cnt(x2);
    while (true) {
        if (f_x1 > f_x2) {
            r.update_range(x1, r.right());
            x1 = x2;
            f_x1 = f_x2;
            x2 = r.left() + golden_const * r.delta();
            f_x2 = func_cnt(x2);
        } else {
            r.update_range(r.left(), x2);
            x2 = x1;
            f_x2 = f_x1;
            x1 = r.right() - golden_const * r.delta();
            f_x1 = func_cnt(x1);
        }
        if (r.delta() < epsilon) {
            break;
        }
    }
    long double f_min, x_min;
    if (f_x1 < f_x2) {
        f_min = f_x1;
        x_min = r.left();
    } else {
        f_min = f_x2;
        x_min = r.right();
    }
    information_search answer(x_min, f_min, cnt, r);
    return answer;
}
\end{lstlisting}


\newpage
\subsection {Метод Фибоначчи} 
\begin{lstlisting}
information_search search_methods::fibonacci(std::function<long double(long double)> &func, range r) {
    size_t cnt;
    std::function<long double(long double)> func_cnt = find_cnt_func(func, cnt);
    auto n = static_cast<size_t>(floorl(log(sqrtl(5) * (r.right() - r.left()) / epsilon) / logl((1 + sqrtl(5)) / 2)));
    long double x1 = r.left() + F(n - 2) / F(n) * r.delta();
    long double x2 = r.left() + F(n - 1) / F(n) * r.delta();
    long double f_x1 = func(x1);
    long double f_x2 = func(x2);
    for (size_t k = 1; k < n - 1; k++) {
        if (f_x1 > f_x2) {
            r.update_range(x1, r.right());
            x1 = x2;
            f_x1 = f_x2;
            x2 = r.left() + F(n - k - 1) / F(n - k) * r.delta();
            if (k != n - 2) {
                f_x2 = func_cnt(x2);
            }
        } else {
            r.update_range(r.left(), x2);
            x2 = x1;
            f_x2 = f_x1;
            if (k != n - 2) {
                x1 = r.left() + F(n - k - 2) / F(n - k) * r.delta();
                f_x1 = func_cnt(x1);
            }
        }
    }
    x2 = x1 + epsilon;
    r.update_range(x1, x2);
    f_x2 = func_cnt(x2);
    information_search answer(x2, f_x2, cnt, r);
    return answer;
}
\end{lstlisting}
	

\newpage
\subsection{Метод парабол} 
\begin{lstlisting}
std::random_device rd;
std::default_random_engine eng(rd());
std::uniform_real_distribution<long double> distribution(0, 1);

long double next_mid(const range &r) {
    return r.left() + distribution(eng) * r.delta();
}

std::pair<long double, long double>
find_mid(const std::function<long double(long double)> &func, const range &r, const long double &f_x1,
         const long double &f_x2) {
    long double mid = 0, f_mid = 0;
    do {
        mid = next_mid(r);
        f_mid = func(mid);
    } while (f_mid >= f_x2 || f_mid >= f_x1);
    return {mid, f_mid};
}
long double search_methods::calc_u(const long double &x1, const long double &f_x1,
                   const long double &x2, const long double &f_x2,
                   const long double &x3, const long double &f_x3) {
    const long double delta_x2_x1 = (x2 - x1);
    const long double delta_f2_f3 = (f_x2 - f_x3);
    const long double delta_x2_x3 = (x2 - x3);
    const long double delta_f2_f1 = (f_x2 - f_x1);
    return x2 - (delta_x2_x1 * delta_x2_x1 * delta_f2_f3 - delta_x2_x3 * delta_x2_x3 * delta_f2_f1) /
                (2 * (delta_x2_x1 * delta_f2_f3 - delta_x2_x3 * delta_f2_f1));
}
\end{lstlisting}
\newpage
\begin{lstlisting}
information_search search_methods::parabolas(std::function<long double(long double)> &func, range r) const {
    size_t cnt = 0;
    const std::function<long double(long double)> func_cnt = find_cnt_func(func, cnt);
    long double x1 = r.left();
    long double x3 = r.right();
    long double f_x1 = func_cnt(x1);
    long double f_x3 = func_cnt(x3);
    long double x_min = 0, f_min = 0;
    auto [x2, f_x2] = find_mid(func_cnt, r, f_x1, f_x3);
    while (true) {
        long double u = calc_u(x1, f_x1, x2, f_x2, x3, f_x3);
        long double f_u = func_cnt(u);
        if (std::abs(u - x2) < epsilon){
            x_min = u;
            f_min = f_u;
            break;
        }
        if (u < x2) {
            if (f_u < f_x2) {
                x3 = x2;
                x2 = u;
                f_x3 = func_cnt(x3);
                f_x2 = func_cnt(x2);
            } else {
                x1 = u;
                f_x1 = func_cnt(x1);
            }
        } else {
            if (f_x2 < f_u) {
                x3 = u;
                f_x3 = func_cnt(x3);
            } else {
                x1 = x2;
                x2 = u;
                f_x1 = func_cnt(x1);
                f_x2 = func_cnt(x2);
            }
        }
        r.update_range(x1, x3);
    }
    information_search answer(x_min, f_min, cnt, r);
    return answer;
}
\end{lstlisting}

\newpage
\subsection{Комбинированный метод Брента}
\begin{lstlisting}
information_search search_methods::combined_brent(std::function<long double(long double)> &func, range r) const {
    size_t cnt = 0;
    const std::function<long double(long double)> func_cnt = find_cnt_func(func, cnt);
    long double a = r.left(), c = r.right();
    long double x = a + golden_const * r.delta();
    long double f_x = func_cnt(x);
    long double w = x, v = x;
    long double f_w = f_x, f_v = f_x;
    long double d = r.delta(), e = r.delta();
    long double x_min = 0, f_min = 0;
    while (true) {
        long double g = e;
        e = d;
        long double tol = epsilon * std::abs(x) + epsilon / 10;
        long double u;
        bool normal_parabola = false;
        if (std::abs(x - (a + c) / 2) + r.delta() / 2 <= 2 * tol){
            x_min = x;
            f_min = f_x;
            break;
        }
        if (x != w && w != v && v != x && f_x != f_w && f_w != f_v && f_v != f_x){
            u = calc_u(x, f_x, w, f_w, v, f_v);
            if (a <= u && u <= c && std::abs(u - x) < g / 2){
                normal_parabola = true;
                if (u - a < 2 * tol || c - u < 2 * tol){
                    u = x - sign(x - (a + c) / 2) * tol;
                }
            }
        }
        if (!normal_parabola){
            if (x < (a + c) / 2){
                u = x + golden_const * (c - x);
                e = c - x;
            } else {
                u = x - golden_const * (x - a);
                e = x - a;
            }
        }
        if (std::abs(u - x) < tol){
            u = x + sign(u - x) * tol;
        }
        d = std::abs(u - x);
        long double f_u = func_cnt(u);
        if (f_u <= f_x){
            if (u >= x){
                a = x;
            } else {
                c = x;
            }
            v = w;
            f_v = f_w;
            w = x;
            f_w = f_x;
            x = u;
            f_x = f_u;
        } else {
            if (u >= x){
                c = u;
            } else {
                a = u;
            }
            if (f_u <= f_w || w == x){
                v = w;
                f_v = f_w;
                w = u;
                f_w = f_u;
            } else {
                if (f_u <= f_v || v == x || v == w){
                    v = u;
                    f_v = f_u;
                }
            }
        }
        r.update_range(a, c);
    }
    information_search answer(x_min, f_min, cnt, r);
    return answer;
}
\end{lstlisting}