\chapter{Методы решения СЛАУ}

\section{Постановка задачи}

\begin{enumerate}
    \item Реализовать методы решения СЛАУ:
    \begin{itemize}
        \item Прямой метод на основе LU-разложения;
        \item Метод Гаусса;
        \item Метод сопряженных градиентов.
    \end{itemize}
    \item Реализовать генераторы для следующих матриц:
    \begin{itemize}
        \item Случайные матрицы;
        \item Случайные матрицы в профильном формате;
        \item Матрицы Гильберта.
    \end{itemize}
    \item Провести исследование раборы реализованных алгоритмов на основе генераторов.
\end{enumerate}

\section{Ход работы}

Для начала мы реализовали класс $MatrixProfileFormat$, в котором содеражтся 4 массива.
Массив $profile$ хранит симметричный профиль данной матрицы в таком формате: $profile[i]$ -- 
число элементов под (над) диагональю, которые входят в профиль. Массивы $al$ и $au$
хранят самы элементы профиля под и над диагональю соответственно. Массив $diag$ 
хранит диагональ матрицы. Также для этого класса мы перегрузили операторы '()' и '*'.
Оператор '()' перегружен для двух параметров, чтобы обращатся к элементам матрицы.
Оператор '*' нужен для умножения матрицы на вектор.

Далее мы сделали класс $LUMartix$, который унаследовался от $MatrixProfileFormat$.
В этом классе мы хранили матрицу в разложенном формате, а в конструкторе написали
само разложение. Реализовали методы $L(i, j)$ и $U(i, j)$ для этого класса, чтобы можно было 
обращатся к разным матрицам этого класса, при этом не задумываясь о том, как хранятся матрицы
внутри этого класса.

Далее сделали генераторы. Мы не оборачивали их в классы, просто сделали набор методов в отдельных файлах.

Затем сделали написалы функции $LUSolver$ и $GaussSolver$, которые принимали на вход 
матрицу из левой части СЛАУ и вектор из правой и решали задачу соответствующим методом.

Аналогично $MatrixProfileFormat$ далее написали класс $MatrixSparseFormat$, но только для разреженного формата
франения матриц. В нем также хранятся $al$, $au$, $profile$, $diag$, а также массив $indexes$, 
который хранит индексы ненулевых элементов. Различия еще в том, что 
в массиве $profile$ хранятся только ненулевые элементы.

\newpage
\section{LU-разложение}

Пусть дана матрица A. Для LU-разложения мы воспользовались вот такими передходами
$$a_{ik} \leftarrow a_{jk} - a_{ik}a_{ji}/a_{ii}, \ j = (i + 1)..n, \ k = (i + 1)..n$$
$$a_{ji} \leftarrow a_{ji}/a_{ii}, \ j = (i + 1)..n$$
Под главной диагонолью хранилась матрица $L$, над диагональю матрица $U$.

Для того, чтобы решить LUx = b, используя LU-разложение, мы вначале решили систему $Ly = b$
используя очевидные формулы:
$$y_1 = b_1$$
$$y_2 = b_2 - y_1l_{21}$$
$$\vdots$$
$$y_n = b_n - \sum_{i = 1}^{n - 1}y_il_{ni}$$


А далее решили СЛАУ $Ux = y$ по аналогичным формулам, только снизу вверх:
$$x_1 = (y_1 - \sum_{i = 2}^{n}x_iu_{1i})/u_{11}$$
$$\vdots$$
$$x_(n - 1) = (y_{n - 1} - x_nu_{n - 1, n})/u_{n - 1, n - 1}$$
$$x_n = y_n/u_{nn}$$

Таблица с результатами ниже.


\resizebox{6.9cm}{!}{
    \csvautotabular[separator=semicolon]{lu/table.csv}
}

\

Также таблица с матрицами Гильберта.

\

\resizebox{6.9cm}{!}{
    \csvautotabular[separator=semicolon]{lu/table_hilbert.csv}
}

Как мы видим в силу плохой обусловленности матриц Гильберта, мы получили большую погрешность.


\newpage
\section{Метод Гаусса}


\newpage
\section{Реализация методов}
